\section{B-Bäume}

\begin{definition}
    \index{B-Baum}

    Ein B-Baum vom Typ $t$ mit $t \geq 2, t \in \mathbb{N}$, ist ein
    gerichteter Baum $T$ mit Wurzel $T.\mathit{root}$, und hat die folgenden
    Eigenschaften:
    \begin{enumerate}
        \item Jeder Knoten $x$ enthält die folgenden Informationen:
            $$ (\underbrace{x.n}_{\mathclap{\textsl{Anzahl der Schlüssel im
            Knoten}}}, x.c_1, x.\mathit{key}_1,
            \overbrace{x.c_2}^{\mathclap{\textsl{Pointer auf Kindknoten}}},
            \underbrace{x.\mathit{key}_2}_{\mathclap{\textsl{Schlüssel}}}
            \dots, x.\mathit{key}_{x.n}, x.c_{x.n+1},
            \overbrace{x.\mathit{leaf}}^{\mathclap{\textsl{Boolean: Knoten ist
            Blatt}}}) $$

        \item $x.\mathit{key}_i \leq x.\mathit{key}_{i+1} \forall i$ \qquad
        (d.h. gleiche Schlüssel sind erlaubt)

        \item Für jeden Schlüssel $k_i$ gilt: steht $k_i$ im Unterbaum mit
        Wurzel $x.c_i$, so gilt: $$k_1 \leq x.\mathit{key}_1 \leq k_2 \leq
        x.\mathit{key}_2 \leq \cdots \leq x.\mathit{key}_{x.n} \leq k_{n+1}
        \qquad (i=1 \dots n+1)$$

        \item Alle Blätter haben die gleiche Tiefe.

        \item Jeder Knoten außer der Wurzel enthält mindestens $t-1$ Schlüssel,
        die Wurzel mindestens $1$ Schlüsel.

        \item Jeder Knoten enthält höchstens $2t-1$ Schlüssel. Bei $2t-1$
        Schlüsseln ist der Knoten voll.
    \end{enumerate}
\end{definition}


\begin{lemma}
    \index{B-Baum!Höhe}

    Ist $n \geq 1$, so gilt für einen B-Baum $T$ vom Typ $t, t \in \mathbb{N},
    t \geq 2$, mit $n$ Schlüsseln: $h(T) \leq \log_t(\frac{n+1}{2})$
\end{lemma}


\begin{beweis}
    Die Wurzel hat mindestens $1$ Schlüssel, die anderen Knoten mindestens
    $t-1$ Schlüssel. Die Anzahl der Pointer auf Kindknoten ist an die Anzahl
    der Schlüssel gebunden. Es gibt (sofern der Baum voll genug ist) deshalb
    mindestens $2$ Knoten der Tiefe $1$, und mindestens $2t$ Knoten der Tiefe
    $2$, $2t^2$ Knoten der Tiefe 3, usw. In Tiefe $n$ gibt es mindestens
    $2t^{n-1}$ Knoten. 
    \begin{align}
        n \geq& \overbrace{1}^{\mathclap{\textsl{Schlüssel in Wurzel}}} + (t-1)
        \overbrace{\sum_{i=1}^n 2t^{i-1}}^{\mathclap{\textsl{andere
        Schlüssel}}} \\
        n \geq& 1+2(t-1) \sum_{i=1}^n t^{i-1} = 1+2(t^n-1) = 2t^n-1 \\
        \frac{n+1}{2} \geq& t^n \qquad \text{($\log$ zur Basis $t$ anwenden)}  \\
        n \leq& \log_t(\frac{n+1}{2})
    \end{align}
    % TODO bild 
\end{beweis}


\begin{bemerkung}
    Ebenfalls Teil der Vorlesung sind die Algorithmen \verb+B-Tree-Search+,
    \verb+B-Tree-Create+, \verb+B-Tree-Split-Child+, \verb+B-Tree-Insert+ und
    \verb+B-Tree-Insert-Nonfull+.
\end{bemerkung}
